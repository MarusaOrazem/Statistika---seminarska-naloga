\documentclass{article}
\usepackage[utf8]{inputenc}
\usepackage{graphicx}
\usepackage{mathtools}
\usepackage{amssymb}
\usepackage{amsmath}
\usepackage{eurosym}



\begin{document}

\section*{1. NALOGA}
\subsection*{a)}
\subsubsection*{Povprečni dohodek}

Populacija je velikosti $43886$. Vzamemo enostavni slučajni vzorec $400$ enot.\\
Sledi: $N = 43886, n = 400$.\\
Naj bo $X_k$ skupni dohodek v k-ti družini.\\
Torej je povprečni dohodek v Kindergradu: 
\begin{equation*}
\overline{X} = \frac{X_1 + ... + X_n}{n}.
\end{equation*}

\subsubsection*{Standardna napaka}
Vemo: $se( \overline{X} ) = \sqrt{var( \overline{X} )}$. Ker imamo enostavni slučajni vzorec, vemo tudi, da je
\begin{equation*}
var( \overline{X} ) = \frac{1}{n} \frac{N-n}{N-1} \sigma^2
\end{equation*}
kjer je $\sigma^2$ populacijska varjanca. Nepristranska cenilka za $\sigma^2$ je
\begin{equation*}
\hat{\sigma}^2 = \frac{N-1}{N(n-1)} \sum_{k=1}^{n}( X_k - \overline{X} )^2.
\end{equation*}
Sledi:
\begin{equation*}
\widehat{se(\overline{X})} = \sqrt{ \frac{1}{n} \frac{N-n}{N(n-1)} \sum_{k=1}^n (X_k - \overline{X} )^2.
}
\end{equation*}

\subsubsection*{Interval zaupanja}
Iz navodil sledi, da je interval zaupanja enak $\overline{X} \pm 1,96 \cdot se( \overline{X} )$.

\subsubsection*{Končne vrednosti}

\subsection*{b)}

Če stratificiramo, mora veljati 
\begin{equation*}
\frac{n_k}{n} = \frac{N_k}{N},   \sum_{k=1}^k n_k = n .
\end{equation*}
V našem primeru startificiramo po četrtih, torej $k=4$.
Vemo:\\
$N_1 = 10 149$ (\textit{severna četrt}),\\
$N_2 = 10 390$ (\textit{vzgodna četrt}),\\
$N_3 = 13 457$ (\textit{južna četrt}),\\
$N_4 = 9 890$ (\textit{zahodna četrt}).
Če malo obrnemo zgornjo enakost, dobimo
\begin{equation*}
n_k = \frac{N_k}{N} n.
\end{equation*}
Izračunamo za $k = 1,2,3,4$ in upoštevamo vrednosti $N_1,N_2,N_3,N_4$ ter $N = 43886$.\\
Dobimo:
\begin{equation*}
n_1 = \frac{10149}{43886} \cdot 400 = 92,5033 \rightarrow n_1 = 92,
\end{equation*}

\begin{equation*}
n_2 = \frac{10390}{43886} \cdot 400 = 94, 699 \rightarrow n_2 = 95,
\end{equation*}

\begin{equation*}
n_3 = \frac{13457}{43886} \cdot 400 = 122,654 \rightarrow n_3 = 123,
\end{equation*}

\begin{equation*}
n_4 = \frac{9890}{43886} \cdot 400 = 90,142 \rightarrow n_4 = 90.
\end{equation*}

Preverimo:
\begin{equation*}
\sum_{k=1}^{4}n_k = 92+96+123+90 = 400.
\end{equation*}

Naj bo sedaj $X_{kj}$ povprečni dohodek j-te družine v k-tem stratumu.\\
Povprečni dohodek družine se sedaj izraža kot:
\begin{equation*}
\overline{X} = \frac{1}{n} \sum_{k=1}^{\# stratumov} \sum_{j=1}^{n_k} X_{kj}.
\end{equation*}
Standarna napaka $se(\overline{X}) = \sqrt{var( \overline{X})}$:
\begin{equation*}
var(\overline{X}) = \sum_k w_k^2 var(\overline{X_k}))  = \sum_k w_k^2 \cdot \frac{\hat{\sigma}_k^2}{n_k} \cdot \frac{N_k-n_k}{N_k-1},
\end{equation*}
kjer je $w_k = \frac{N_k}{N}$ delež, $\sigma_k^2$ pa populacijska varjanca v k-tem stratumu. Torej:
\begin{equation*}
\hat{\sigma}_k^2 = \frac{N_k-1}{N_k(n_k-1)} \sum_{j=1}^{n_k} (X_{kj}-\overline{X_k})^2,
\end{equation*}
kjer je $X_k$ povprečje k-tega stratuma.\\
Interval zaupanja: $\overline{X} \pm 1,96 \cdot se(\overline{X})$. \\
Vstavimo podatke in dobimo:


\subsection*{c)}
Variance znotraj četrti:\\
Variance med četrtmi dobimo tako, da izračunamo povprečje dohodka v vsaki četrti, jo primerno obtežimo in izračunamo varjanco. Torej dobimo:

\section*{2. NALOGA}

\section*{3.NALOGA}
\subsection*{a)}
\begin{equation*}
X\sim f(x,\alpha) =\begin{cases}
 \frac{\Gamma(3\alpha)}{\Gamma(\alpha)\Gamma(2\alpha)} x^{\alpha-1}(1-x)^{2\alpha-1}, & 0<x<1.\\
0, & sicer.
\end{cases}
\end{equation*}
Vemo: 
\begin{equation*}
E(X) = \frac{1}{3}, var(X) = \frac{2}{9(3\alpha+1)}.
\end{equation*}
Ker je:
\begin{equation*}
var(X)=E(X^2)-E(X)^2,
\end{equation*}
sledi: 
\begin{equation*}
E(X^2) = var(X) + E(X)^2 = \frac{2}{9(3\alpha+1)} + \frac{1}{9} = 
\frac{\alpha +1}{3(3\alpha+1)}
\end{equation*}
$E(X^2)$ je drugi moment slučajne spremenljivke X, torej:
\begin{equation*}
E(X^2) = \frac{1}{n} \sum_{i=1}^n X_i^2.
\end{equation*}
Zgornji enačbi izenačimo in poračunamo $\alpha$:
\begin{equation*}
 \frac{1}{n} \sum_{i=1}^n X_i^2 = \frac{\alpha +1}{3(3\alpha+1)}
\end{equation*}
\begin{equation*}
\frac{9\alpha+3}{n} \sum_{i=1}^n X_i^2 = \alpha +1
\end{equation*}
\begin{equation*}
\frac{9\alpha}{n} \sum_{i=1}^n X_i^2  - \alpha = 1 - \frac{3}{n}\sum_{i=1}^n X_i^2
\end{equation*}
\begin{equation*}
\alpha = \frac{1 - \frac{3}{n}\sum_{i=1}^n X_i^2}{\frac{9}{n} \sum_{i=1}^n X_i^2  - 1}
\end{equation*}

\subsection*{b)}
\begin{equation*}
 f_X(x,\alpha) =\begin{cases}
 \frac{\Gamma(3\alpha)}{\Gamma(\alpha)\Gamma(2\alpha)} x^{\alpha-1}(1-x)^{2\alpha-1}, & 0<x<1.\\
0, & sicer.
\end{cases}
\end{equation*}

\begin{equation*}
L_1(\alpha|x) =  \frac{\Gamma(3\alpha)}{\Gamma(\alpha)\Gamma(2\alpha)} x^{\alpha-1}(1-x)^{2\alpha-1}
\end{equation*}

\begin{equation*}
\begin{split}
L(\alpha|x_1,...,x_n) =  L_1(\alpha|x_1) \cdot ... \cdot L_1(\alpha|x_n) =             \\ 
 \left( \frac{\Gamma(3\alpha)}{\Gamma(\alpha)\Gamma(2\alpha)} \right)^n x_1^{\alpha-1} \cdot ... \cdot x_n^{\alpha-1} (1-x_1)^{2\alpha-1} \cdot ... \cdot (1-x_n)^{2\alpha-1}
\end{split}
\end{equation*}

\begin{equation*}
\begin{split}
l(\alpha|x_1,...,x_n) = ln( L(\alpha|x_1,...,x_n)) = 
l_1(\alpha|x_1) + ... + l_1(\alpha|x_n)
\end{split}
\end{equation*}

\begin{equation*}
\begin{split}
l_1(\alpha|x) = ln(L_1(\alpha|x)) =\\ ln(\Gamma(3\alpha)) - ln(\Gamma(\alpha)) - ln(\Gamma(2\alpha)) + (\alpha-1)ln(x) + (2\alpha-1)ln(1-x)
\end{split}
\end{equation*}

\begin{equation*}
\begin{split}
l(\alpha|x) =\\ \sum_{i=1}^n\left( ln(\Gamma(3\alpha)) - ln(\Gamma(\alpha)) - ln(\Gamma(2\alpha)) + (\alpha-1)ln(x_i) + (2\alpha-1)ln(1-x_i) \right)
\end{split}
\end{equation*}

\begin{equation*}
\frac{\partial l}{\partial \alpha} = \sum_{i = 1}^n \frac{1}{\Gamma(3\alpha)} \Gamma^{'}(3\alpha) 3 -
\frac{1}{\Gamma(\alpha)} \Gamma^{'}(\alpha) - \frac{1}{\Gamma(2\alpha)} \Gamma^{'}(2\alpha) 2 +
ln(x_i) + 2ln(1-x_i)
\end{equation*}

\begin{equation*}
\begin{split}
\sum_{i = 1}^n \frac{1}{\Gamma(3\alpha)} \Gamma^{'}(3\alpha) 3 -
\frac{1}{\Gamma(\alpha)} \Gamma^{'}(\alpha) - \frac{1}{\Gamma(2\alpha)} \Gamma^{'}(2\alpha) 2 =
ln(\frac{1}{x_i(1-x_i)})
\end{split}
\end{equation*}
Cenilka obstaja, ko ima zgornja enačba rešitev.\\
\textit{Uporabimo funkcijo digamma in rešimo do konca? no clue.}

\subsection*{c)}
$var(\hat{\alpha}) = \frac{1}{nI_1(\hat{\alpha})}$.
\begin{equation*}
I_1(\hat{\alpha}) = -E\left[  \frac{\partial^2l_1(\alpha|x)}{\partial \alpha^2}   \right]
\end{equation*}

\begin{equation*}
\begin{split}
\frac{\partial^2l_1(\alpha|x)}{\partial \alpha^2} =\\  3\frac{\Gamma^{''}(3\alpha)\Gamma(3\alpha) - \Gamma^{'}(3\alpha)^2}{\Gamma(3\alpha)^2} -
\frac{\Gamma^{''}(\alpha)\Gamma(\alpha) - \Gamma^{'}(\alpha)^2}{\Gamma(\alpha)^2} -
2\frac{\Gamma^{''}(2\alpha)\Gamma(2\alpha) - \Gamma^{'}(2\alpha)^2}{\Gamma(2\alpha)^2}
\end{split}
\end{equation*}

\begin{equation*}
var(\hat{\alpha}) =\frac{1}{n} \frac{1}{ 3\frac{\Gamma^{''}(3\alpha)\Gamma(3\alpha) - \Gamma^{'}(3\alpha)^2}{\Gamma(3\alpha)^2} -
\frac{\Gamma^{''}(\alpha)\Gamma(\alpha) - \Gamma^{'}(\alpha)^2}{\Gamma(\alpha)^2} -
2\frac{\Gamma^{''}(2\alpha)\Gamma(2\alpha) - \Gamma^{'}(2\alpha)^2}{\Gamma(2\alpha)^2}}
\end{equation*}


\section*{4.NALOGA}
Za reševanje te naloge, sem si pomagala s knjigo \textit{Mathematical statistics and data analysis}. Iz poglavja 9.5, na strani 341, izvemo, da če imamo model, kjer so verjetnosti vektorsko porazdeljene in imamo opis za hipotezo $H_0$, kjer je $p = p(\theta)$, kjer je $\theta$ neznam, je števec za verjetnostno funkcijo enak:
\begin{equation*}
\max_{p\in \omega_0} \left(  \frac{n!}{x_1!...x_m!} \right) p_1(\theta)^{x_1} \cdots p_m(\theta)^{x_m},
\end{equation*}
kjer so $x_i$ opazovane vrednosti v $m$ celicah.
Maksimum bo dosežen pri:
\begin{equation*}
\hat{p_i} =  \frac{x_i}{n}.
\end{equation*}
Razmerje verjetij je tako enako:
\begin{equation*}
\Lambda= \frac{\frac{n!}{x_1!...x_m!}  p_1(\hat{\theta})^{x_1} \cdots p_m(\hat{\theta})^{x_m}}{\frac{n!}{x_1!...x_m!}  \hat{p_1}^{x_1} \cdots \hat{p_m}^{x_m}}.
\end{equation*}

Poglejmo sedaj naš primer. Podatke imamo podane v razpredelnici za $m=0,1,...,12.$, $n = \text{\# podatkov} = 6115.$ Oprazovane vrednosti $x_1,...,n_{12}$ prebermo iz drugega stolpca v tabeli. Računamo po postopku opisanem zgoraj:
\begin{equation*}
\max \left( \frac{6115!}{7! 45! 181! \cdots 24!3!}  \right) p_1^7 p_2^{45}\cdots p_{12}^3
\end{equation*}
Maksimum je torej dosežen pri:
\begin{equation*}
\hat{p}_1 = \frac{7}{6115}, \hat{p}_2 = \frac{45}{6115}, \cdots, \hat{p}_{12} = \frac{3}{6115}, 
\end{equation*}
Poračunati moramo še imenovalec ulomka, da bomo dobili razmerje verjetij. Testirati želimo hipotezo, da je število moških potomcov, ki se rodijo v družini z 12 otroci, porazdeljeno binomsko $Bin(12,p)$. Torej je verjetnostna funkcija L enaka:
\begin{equation*}
L = \left( \frac{6115!}{7! 45! 181! \cdots 24!3!}  \right) \left( (1-p)^{12} \right)^7 \left( 2p(1-p)^{11}\right)^{45} \cdots \left( p^{12}\right)^3.
\end{equation*}
Izraz logaritmiramo in odvajamo po p, izračunamo maksimum.
\begin{equation*}
\frac{\partial l}{\partial p} = 0.
\end{equation*}
Po krajšem izračunu dobimo:
\begin{equation*}
\frac{35280}{p} = \frac{38100}{1-p}
\end{equation*}
Tako dobimo:
\begin{equation*}
\hat{p} = \frac{35280}{73380} = 0.4807849550286182.
\end{equation*}
Torej po Wilkinsonovem izreku zapišemo:
\begin{equation*}
\lambda = 2ln(\Lambda) = 2 l(\hat{p}_0,...,\hat{p}_{12}) - 2l(\hat{p})
\end{equation*}
Vstavimo zgoraj poračunane vrednosti:
\begin{equation*}
2\left( 7ln(\hat{p}_1) + 45ln(\hat{p}_2)\cdots 3ln(\hat{p}_{12}) \right)- 2 \left( 7*12ln(1-\hat{p}) + 45(ln(12\hat{p})+11ln(1-\hat{p})) + \cdots + 3*12ln(\hat{p})\right)
\end{equation*}
Ko vstavimo podatke dobimo:
\begin{equation*}
\lambda = 97.0065.
\end{equation*}
Wilksov izrek nam pove, da bomo hipotezo zavrnili, če je $\lambda > \chi^2_{1-\alpha}(dim\Omega - dim\Omega_0)$.
Dimenziji prostorov sta očitni: $dim\Omega - dim\Omega_0 = 12-1 = 11$. Testitamo za $\alpha = 0.01 $ in $\alpha = 0.05$. Iz tabele preberemo vrednosti in dobimo:
\begin{equation*}
\chi^2_{0.99}(11) = 19.68, \chi^2_{0.95}(11) = 24.72.
\end{equation*}
Vidimo, da sta vrednosti v obeh primerih manjši od $\lambda$, zato hipotezo v obeh primerih zavrnemo.\\
Model bi lahko bil napačen iz več razlogov. Eden bi bil naprimer, da nismo upoštevali vsa rojstva, vendar le preživele otroke. Lahko, da smo zbrali podatke v različnih zgodovinskih obdobjih, katera vplivajo na rojstva otrok (vojna in podobno).

\section*{5.NALOGA}
\subsubsection*{Ocen po metodi največjega verjetja}
Da bomo vedeli kako so porazdeljeni $Y_i$-ji, poračunamo njihovo pričakovano vrednost in varjanco.
\begin{equation*}
E(Y_i) = E(\beta_0+\beta_1x_i+\epsilon_i) = \beta_0+\beta_1x_1
\end{equation*}

\begin{equation*}
var(Y_i) = \sigma^2
\end{equation*}

Torej so $Y_i$ porazdeljeni $N( \beta_0+\beta_1x_1, \sigma^2)$.
Dobimo:
\begin{equation*}
L_1 = \frac{1}{\sqrt{2\pi \sigma^2}}e^{-{\frac{(y_i - \beta_o-\beta_1x_i)^2}{2\sigma^2}}}
\end{equation*}
Verjetnostna funkcija je sestavljena iz produktov porazdelitvenih funkcij. Torej dobimo:
\begin{equation*}
log(L) = -nlog(\sigma) - \frac{n}{2}log(2\pi)-\frac{1}{2\sigma^2}\sum_{i=1}^n (y_i-\beta_o-\beta_1x_i)^2
\end{equation*}.
Najprej poračunamo za prvi parameter $\beta_0$. Enačbo odvajamo po $\beta_0$ in enačimo z 0.

\begin{equation*}
\begin{split}
\frac{\partial}{\partial\beta_0} log(L) = \frac{\partial}{\partial\beta_0} (  -nlog(\sigma) - \frac{n}{2}log(2\pi)-\frac{1}{2\sigma^2}\sum_{i=1}^n (y_i-\beta_o-\beta_1x_i)^2)=\\
= -\frac{1}{\sigma^2}\sum_{i=1}^n(y_i-\beta_o-\beta_1x_i)
\end{split}
\end{equation*}
Dobimo:
\begin{equation*}
\sum_{i=1}^n(y_i-\beta_o-\beta_1x_i) = 0
\end{equation*}
Sledi:
\begin{equation*}
\sum_{i=1}^n y_i - n\beta_0 -\beta_1\sum_{i=1}^nx_i = 0
\end{equation*}
Torej je končni rezultat enak:
\begin{equation*}
\beta_o = \frac{1}{n} \sum_{i=1}^n y_i - \beta_1\sum_{i=1}^nx_i.
\end{equation*}

Postopek ponovimo za drugi parameter $\beta_1$:
\begin{equation*}
\begin{split}
\frac{\partial}{\partial\beta_1} log(L) = \frac{\partial}{\partial\beta_1} (  -nlog(\sigma) - \frac{n}{2}log(2\pi)-\frac{1}{2\sigma^2}\sum_{i=1}^n (y_i-\beta_o-\beta_1x_i)^2)=\\
= -\frac{1}{\sigma^2}\sum_{i=1}^n x_i(y_i-\beta_0-\beta_1x_i)
\end{split}
\end{equation*}

Dobimo:
\begin{equation*}
\sum_{i=1}^nx_i(y_i-\beta_o-\beta_1x_i) = 0.
\end{equation*}

Sledi:
\begin{equation*}
\sum_{i=1}^nx_iy_i-n\beta_o-\beta_1\sum_{i=1}^nx_i = 0.
\end{equation*}

\begin{equation*}
\sum_{i=1}^nx_iy_i-n\beta_o-\beta_1\sum_{i=1}^nx_i = 0.
\end{equation*}
Vstavimo vrednost za $\beta_0$:
\begin{equation*}
\sum_{i=1}^nx_iy_i-n\left(\frac{1}{n} \sum_{i=1}^n y_i - \beta_1\sum_{i=1}^nx_i\right)-\beta_1\sum_{i=1}^nx_i = 0.
\end{equation*}
\begin{equation*}
\sum_{i=1}^nx_iy_i- \sum_{i=1}^n y_i - \beta_1\left(n\sum_{i=1}^nx_i-\sum_{i=1}^nx_i\right) = 0.
\end{equation*}


Torej je končni rezultat enak:
\begin{equation*}
\beta_1 = \frac{\sum_{i=1}^nx_iy_i- \sum_{i=1}^n y_i}{n\sum_{i=1}^nx_i-\sum_{i=1}^nx_i}.
\end{equation*}

Vstavimo še za $\beta_0$:
\begin{equation*}
\beta_o = \frac{1}{n} \sum_{i=1}^n y_i -  \frac{\sum_{i=1}^nx_iy_i- \sum_{i=1}^n y_i}{n\sum_{i=1}^nx_i-\sum_{i=1}^nx_i}\sum_{i=1}^nx_i.
\end{equation*}
\begin{equation*}
\beta_o = \frac{1}{n} \sum_{i=1}^n y_i -  \frac{\sum_{i=1}^nx_i\sum_{i=1}^nx_iy_i- \sum_{i=1}^nx_i\sum_{i=1}^n y_i}{n\sum_{i=1}^nx_i-\sum_{i=1}^nx_i}.
\end{equation*}
\begin{equation*}
\beta_o = \frac{\sum_{i=1}^nx_i\sum_{i=1}^ny_i - \frac{1}{n}\sum_{i=1}^nx_i\sum_{i=1}^ny_i    -\sum_{i=1}^nx_i\sum_{i=1}^nx_iy_i+ \sum_{i=1}^nx_i\sum_{i=1}^n y_i}{n\sum_{i=1}^nx_i-\sum_{i=1}^nx_i}.
\end{equation*}


\subsubsection*{Ocena po metodi najmanjših kvadratov}
Ker so šumi $\epsilon_i$ neodvisni za vsak $i = 1,2,...,n$ in $\epsilon_i \sim N(0,\sigma^2)$ lahko uporabimo izrek Gauss-Markova za ocene parametrov $\beta_0$ in $\beta_1$.
Imamo:
\begin{equation*}
\begin{bmatrix}
y_1\\
\vdots\\
y_n
\end{bmatrix}
=
\begin{bmatrix}
1 & x_1\\
\vdots &\vdots \\
1 & x_n
\end{bmatrix}
\begin{bmatrix}
\beta_0\\
\beta_1
\end{bmatrix}
+
\begin{bmatrix}
\epsilon_1\\
\vdots\\
\epsilon_n
\end{bmatrix}
\end{equation*}

Po izreku Gauss-Markova vemo da je najboljša cenilka za $\beta = (X^TX)^{-1}X^TY$. Računamo:

\begin{equation*}
\left(
\begin{bmatrix}
1 & \cdots & 1\\
x_1 & \cdots & x_n
\end{bmatrix}
\begin{bmatrix}
1 & x_1\\
\vdots &\vdots \\
1 & x_n
\end{bmatrix}
\right)^{-1}
\begin{bmatrix}
1 & \cdots & 1\\
x_1 & \cdots & x_n
\end{bmatrix}
\begin{bmatrix}
y_1\\
\vdots\\
y_n
\end{bmatrix}
=
\end{equation*}

\begin{equation*}
=
\left(
\begin{bmatrix}
n & \sum_{i=1}^nx_i\\
\sum_{i=1}^nx_i & \sum_{i=1}^nx_i^2
\end{bmatrix}
\right)^{-1}
\begin{bmatrix}
\sum_{i=1}^ny_i\\
\sum_{i=1}^nx_iy_i
\end{bmatrix}
=
\end{equation*}

\begin{equation*}
=
\frac{1}{n\sum_{i=1}^nx_i^2-\sum_{i=1}^nx_i\sum_{i=1}^nx_i}
\begin{bmatrix}
\sum_{i=1}^nx_i^2 & -\sum_{i=1}^nx_i\\
-\sum_{i=1}^nx_i & n
\end{bmatrix}
\begin{bmatrix}
\sum_{i=1}^ny_i\\
\sum_{i=1}^nx_iy_i
\end{bmatrix}
=
\end{equation*}

\begin{equation*}
=
\frac{1}{n\sum_{i=1}^nx_i^2-\sum_{i=1}^nx_i\sum_{i=1}^nx_i}
\begin{bmatrix}
\sum_{i=1}^nx_i^2\sum_{i=1}^ny_i - \sum_{i=1}^nx_i\sum_{i=1}^nx_iy_i\\
-\sum_{i=1}^nx_i\sum_{i=1}^ny_i+n\sum_{i=1}^nx_iy_i
\end{bmatrix}
\end{equation*}

Z malo računanja in spretnosti vidimo da sta cenilki enaki.







































\end{document}