\documentclass{article}
\usepackage[utf8]{inputenc}
\usepackage{graphicx}
\usepackage{mathtools}
\usepackage{amssymb}
\usepackage{amsmath}
\usepackage{eurosym}

\begin{document}

\section*{1. NALOGA}
\subsection*{a)}
\subsubsection*{Povprečni dohodek}

Populacija je velikosti $43886$. Vzamemo enostavni slučajni vzorec $400$ enot.\\
Sledi: $N = 43886, n = 400$.\\
Naj bo $X_k$ skupni dohodek v k-ti družini.\\
Torej je povprečni dohodek v Kindergradu: 
\begin{equation}
\overline{X} = \frac{X_1 + ... + X_n}{n}.
\end{equation}

\subsubsection*{Standardna napaka}
Vemo: $se( \overline{X} ) = \sqrt{var( \overline{X} )}$. Ker imamo enostavni slučajni vzorec, vemo tudi, da je
\begin{equation}
var( \overline{X} ) = \frac{1}{n} \frac{N-n}{N-1} \sigma^2
\end{equation}
kjer je $\sigma^2$ populacijska varjanca. Nepristranska cenilka za $\sigma^2$ je
\begin{equation}
\hat{\sigma}^2 = \frac{N-1}{N(n-1)} \sum_{k=1}^{n}( X_k - \overline{X} )^2.
\end{equation}
Sledi:
\begin{equation}
\widehat{se(\overline{X})} = \sqrt{ \frac{1}{n} \frac{N-n}{N(n-1)} \sum_{k=1}^n (X_k - \overline{X} )^2.
}
\end{equation}

\subsubsection*{Interval zaupanja}
Iz navodil sledi, da je interval zaupanja enak $\overline{X} \pm 1,96 \cdot se( \overline{X} )$.

\subsubsection*{Končne vrednosti}

\subsection*{b)}

Če stratificiramo, mora veljati 
\begin{equation}
\frac{n_k}{n} = \frac{N_k}{N},   \sum_{k=1}^k n_k = n .
\end{equation}
V našem primeru startificiramo po četrtih, torej $k=4$.
Vemo:\\
$N_1 = 10 149$ (\textit{severna četrt}),\\
$N_2 = 10 390$ (\textit{vzgodna četrt}),\\
$N_3 = 13 457$ (\textit{južna četrt}),\\
$N_4 = 9 890$ (\textit{zahodna četrt}).
Če malo obrnemo zgornjo enakost, dobimo
\begin{equation}
n_k = \frac{N_k}{N} n.
\end{equation}
Izračunamo za $k = 1,2,3,4$ in upoštevamo vrednosti $N_1,N_2,N_3,N_4$ ter $N = 43886$.\\
Dobimo:
\begin{equation}
n_1 = \frac{10149}{43886} \cdot 400 = 92,5033 \rightarrow n_1 = 92,
\end{equation}

\begin{equation}
n_2 = \frac{10390}{43886} \cdot 400 = 94, 699 \rightarrow n_2 = 95,
\end{equation}

\begin{equation}
n_3 = \frac{13457}{43886} \cdot 400 = 122,654 \rightarrow n_3 = 123,
\end{equation}

\begin{equation}
n_4 = \frac{9890}{43886} \cdot 400 = 90,142 \rightarrow n_4 = 90.
\end{equation}

Preverimo:
\begin{equation}
\sum_{k=1}^{4}n_k = 92+96+123+90 = 400.
\end{equation}

Naj bo sedaj $X_{kj}$ povprečni dohodek j-te družine v k-tem stratumu.\\
Povprečni dohodek družine se sedaj izraža kot:
\begin{equation}
\overline{X} = \frac{1}{n} \sum_{k=1}^{\# stratumov} \sum_{j=1}^{n_k} X_{kj}.
\end{equation}
Standarna napaka $se(\overline{X}) = \sqrt{var( \overline{X})}$:
\begin{equation}
var(\overline{X}) = \sum_k w_k^2 var(\overline{X_k}))  = \sum_k w_k^2 \cdot \frac{\hat{\sigma}_k^2}{n_k} \cdot \frac{N_k-n_k}{N_k-1},
\end{equation}
kjer je $w_k = \frac{N_k}{N}$ delež, $\sigma_k^2$ pa populacijska varjanca v k-tem stratumu. Torej:
\begin{equation}
\hat{\sigma}^2 = \frac{N_k-1}{N_k(n_k-1)} \sum_{j=1}^{n_k} (X_{kj}-\overline{X_k})^2,
\end{equation}
kjer je $X_k$ povprečje k-tega stratuma.\\
Interval zaupanja: $\overline{X} \pm 1,96 \cdot se(\overline{X})$. \\
Vstavimo podatke in dobimo:


\subsection*{c)}




























\end{document}